\documentclass[10pt, conference]{IEEEtran}

\usepackage{cite}
\usepackage{listings}
\usepackage{lipsum}
\usepackage[T1]{fontenc}
\usepackage[portuguese]{babel}

\begin{document}

\title { 
	Category Theory \\
	Arrows e Monads \\
	em Java 
}

\author{

	\IEEEauthorblockN{ Mozart L. Siqueira }
	\IEEEauthorblockA{
		Ci\^encias da Computa\c{c}\~ao \\
		Centro Universit\'ario La Salle - Unilasalle \\
		Email: mozarts@unilasalle.edu.br
	}

	\and	
	
	\IEEEauthorblockN { Pablo M. Parada }
	\IEEEauthorblockA { 
		Ci\^encias da Computa\c{c}\~ao \\
		Centro Universit\'ario La Salle - Unilasalle \\
		Email: pablo.paradabol@gmail.com
	}
}
					  
\maketitle

\begin{abstract}

<escrever>

\end{abstract}

\begin{IEEEkeywords}

<escrever>

\end{IEEEkeywords}

\section{INTRODU\c{C}\~AO}
\label{sec:intro}

<escrever>

\section{SOBRE O PARADIGMA FUNCIONAL}


O paradigma funcional surge da id\'eia de expressar computa\c{c}\~oes atrav\'es de fun\c{c}\~oes combinadas em express\~oes. Assim, fun\c{c}\~oes expressam o que dever\'a ser computado, ao inv\'es de como sera computado \cite{louden2011programming}. Programas s\~ao constru\'idos atrav\'es da composi\c{c}\~ao, tal que fun\c{c}\~oes triviais (ou \textit{building blocks}) s\~ao combinadas dando origem a novas fun\c{c}\~oes que descrevem computa\c{c}\~oes mais complexas.

\textit{Building blocks} n\~ao devem fazer uso de vari\'aveis que dependam de estado, isso significa que a computa\c{c}\~ao deve ser pura e sem efeitos indesejados (ou \textit{side-effects}). Tamb\'em destaca-se o princ\'ipio de imutabilidade, onde o valor \'e de uma vari\'avel \'e determinado em sua cria\c{c}\~ao, n\~ao permitindo novas atribui\c{c}\~oes posteriormente.

\'E poss\'ivel afirmar que ao expressar um programa em uma linguagem funcional obtemos uma maneira concisa de solucionar problemas, dado que este constitui-se de opera\c{c}\~oes e objetos at\^omicos e regras gerais para sua composi\c{c}\~ao \cite{michaelson2011introduction}.

\bibliographystyle{IEEEtran}
\bibliography{references}

\end{document}