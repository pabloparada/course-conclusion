\documentclass[10pt, conference]{IEEEtran}

\usepackage[numbers,sort&compress,square]{natbib}
\usepackage{listings}
\usepackage{lipsum}
\usepackage[T1]{fontenc}
\usepackage[portuguese]{babel}
\usepackage[utf8]{inputenc}
\usepackage{listings}
\usepackage[svgnames]{xcolor}
\usepackage{textcomp}
\usepackage{enumitem}
\usepackage{tikz}
\usetikzlibrary{matrix}

% configs listings
\renewcommand{\lstlistingname}{Listagem}
\lstset{
	basicstyle = \footnotesize\tt,
	breakatwhitespace = true,
	breaklines = true,
	captionpos = b,
	extendedchars = true,
	language = Java,
	frame = single,
	keywordstyle = \bf,
	showspaces = false,
	showstringspaces = false,
	showtabs = false,
	tabsize = 2,
	abovecaptionskip = 1em,
	belowcaptionskip = 1em,
	aboveskip = 2em,
    belowskip = 2em,
    xleftmargin = 5pt,
	xrightmargin = 5pt
}

% configs footnotes
\renewcommand\footnoterule{\kern-3pt \hrule width 2in \kern 2.6pt}

\begin{document}
\bstctlcite{IEEEexample:BSTcontrol}

\title { 
	Category Theory \\
	Arrows e Monads \\
	em Java 
}

\author{

	\IEEEauthorblockN{ Mozart L. Siqueira }
	\IEEEauthorblockA{
		Ciências da Computação \\
		Centro Universitário La Salle - Unilasalle \\
		Email: mozarts@unilasalle.edu.br
	}

	\and	
	
	\IEEEauthorblockN { Pablo M. Parada }
	\IEEEauthorblockA { 
		Ciências da Computação \\
		Centro Universitário La Salle - Unilasalle \\
		Email: pablo.paradabol@gmail.com
	}
}
					  
\maketitle

\thispagestyle{plain}
\pagestyle{plain}

\begin{abstract}

\textcolor{red}{<escrever>}

\end{abstract}

\begin{IEEEkeywords}

\textcolor{red}{<escrever>}

\end{IEEEkeywords}

\section{INTRODUÇÃO}
\label{sec:intro}

<escrever>

\section{SOBRE O PARADIGMA FUNCIONAL}
\label{sec:func-para}
Influenciado principalmente pelo desenvolvimento do lambda calculus \cite{hudak1989conception}, compondo o grupo da programação declarativa, o paradigma funcional utiliza-se da ideia de expressar computações através de funções combinadas em expressões. Neste, funções determinam o que deverá ser computado, ao invés de como sera computado \cite{louden2011programming}. Programas são construídos através da composição, tal que funções triviais (ou building blocks) são combinadas dando origem a novas funções que descrevem computações mais complexas.

Building blocks não devem fazer uso de variáveis que dependam de estado, isso significa que a computação deve ser pura e sem efeitos indesejados (ou side-effects). Também destaca-se o princípio de imutabilidade, onde o valor é de uma variável é determinado em sua criação, não permitindo novas atribuições posteriormente.

É possível afirmar que ao expressar um programa em uma linguagem funcional, obtém-se uma maneira concisa de solucionar problemas, dado que este constitui-se de operações e objetos atômicos e regras gerais para sua composição \cite{michaelson2011introduction}. Estas qualidades são apreciadas nos tempos atuais, onde há necessidade de tratar os problemas oriundos do não-determinismo. Assim, o paradigma funcional mostra-se capaz, inclusive de influenciar outras linguagens como Java \cite{jsr335}.

\subsection{Lambda Expressions e Anonymous Inner Classes}
Ao fornecer funções de primeira classe (também lambda expressions ou closures)\footnote{Lambda expression é uma função que não exige vínculos de classe, como exemplo podendo ser atribuída a uma variável. Com esta caraterística, uma função atua como dado, ou seja, pode ser passada como argumento para outras funções.}, a linguagem Java habilita a substituição de annonymous inner classes (AIC) de forma transparente. Contudo, apesar destas expressarem o mesmo comportamento a nível de código, ambas funcionalidades possuem diferentes implementações sob a Máquina Virtual Java (JVM). Conforme a listagem~\ref{lst:array-aic-sort}, a ordenação de inteiros pode ser implementada a partir de uma AIC em conjunto do método sort da classe Arrays.

\begin{lstlisting}[caption={Sort - Anonymous Inner Class}, label={lst:array-aic-sort}]
Integer[] integers = new Integer[]{5, 4, 3, 2, 1};

Arrays.sort(integers, new Comparator<Integer>() {
    public int compare(Integer a, Integer b) {
        return a.compareTo(b);
    }
});
\end{lstlisting}

Entretanto, o mesmo método pode ser simplificado por uma lambda expression conforme demonstrado na listagem~\ref{lst:array-lambda-sort}. Nesta abordagem obtém-se o mesmo resultado sem os encargos impostos por AIC.

\begin{lstlisting}[caption={Sort - Lambda Expressions}, label={lst:array-lambda-sort}]
Arrays.sort(integers, (a, b) -> a.compareTo(b));
\end{lstlisting}

Conforme afirmou-se anteriormente, AIC e closures de fato diferem sob a JVM. AIC são compiladas, ou seja, geram novos arquivos contendo declarações de classes. Além do mais, ao utilizar a palavra reservada \textbf{this} referencia-se a própria instância anônima. Como representam instâncias de uma classe, estas devem ser carregadas pelo classloader e seus construtores invocados pela máquina virtual. Ambas etapas consomem memória, tanto heap \cite{hunt2011java} para alocação de objetos quanto permgem\footnote{Área de memória limitada separada da heap chamada Permanent Generation que possui a função de armazenar objetos de geração permanente como metadados, classes e métodos.}.

Diferentemente de AIC, lambdas postergam a estratégia de compilação para em tempo de execução, utilizando a instrução invokedynamic \cite{goetz2012translation}. Funções são traduzidas para métodos estáticos vinculados ao arquivo da classe correspondente a sua declaração, eliminando o consumo de memória. Agora, ao referir-se a \textbf{this}, a classe que delimita a lamda expression é acessada, ao contrário de AIC que acessa sua própria instância. Por fim, closures fornecem formas mais expressivas de representar comportamentos.

\section{CATEGORY THEORY E SUAS APLICAÇÕES}
A Category Theory (CT) foi inventada no início dos anos 40 por Samuel Eilenberg e Sunders Mac Lane \cite{eilenbergmaclane1945naturalequivalences} como uma ponte entre os diferentes campos da topologia e algebra \cite{spivak2014cts}. Afim de demonstrar as relações entre estruturas e sistemas matemáticos \cite{maclane1971mat}, a CT estabelece uma linguagem formal capaz de encontrar aplicabilidade em várias áreas da ciência. Tal como Group Theory\footnote{A Group Theory estuda as estruturas algébricas chamadas Group. Um Group é um conjunto de elementos finito ou infinito associado a uma operação binária \cite{spivak2014cts}, como por exemplo a adição ou multiplicação.} abstrai a ideia do sistema de permutações como simetrias de um objeto geométrico, a CT manifesta-se como um sistema de funções entre conjuntos de objetos \cite{awodey2010category}.

\begin{figure}[h]
\centering
	\begin{tikzpicture}[baseline=-0.8ex]
    	\matrix (m) [
        	    matrix of math nodes,
            	row sep=3em,
	            column sep=5em,
				text height=2ex, text depth=0.5ex
            	] {
			A & B \\
			  & C \\
        };
    	\path[->]
        	(m-1-1) edge node[above] {$f$} (m-1-2)
			(m-1-2) edge node[right] {$g$} (m-2-2)
			(m-1-1) edge node[left = 0.9cm, below = 0.001cm] {$h =$ $g \circ f$} (m-2-2);
	\end{tikzpicture}
\caption{Sitema de funções e coleções de objetos}
\label{fig:f-g-composition}
\end{figure}

Conforme a figura~\ref{fig:f-g-composition}, os conjuntos de objetos são representado por $A$, $B$ e $C$. Nesta mesma estrutura, as funções $f$ e $g$ denotam os morfismos entre os diferentes conjuntos de objetos, tal que $f: A \rightarrow B$ e $g: B \rightarrow C$. Temos também a função $h$, definida por $h: A \rightarrow C$ como produto da composição de $f$ e $g$. A composição de funções, conforme a seção~\ref{sec:func-para} e expressa através de $h$ na figura~\ref{fig:f-g-composition}, firma uma ligação entre o paradigma funcional e a organização estrutural derivada da CT.

A partir desta ligação, Moggi em seus trabalhos \cite{moggi1989lambdaandmonads, mogi1991notionsofcompandmonad} introduziu o conceito de monad em Haskell. Este foi capaz de fornecer os mecanismos necessários para capacitar a solução de um dos problemas fundamentais da linguagem,  pertencente ao conjunto The Awkward Squad \cite{jones2001tacklingthe}. A partir de outros conceitos como functors, categories, arrows, o trabalho de Moggi foi incrementado criando novas estruturas que estão presentes nas versões mais recentes da linguagem.

\subsection{Category -- Objetos e Morfismos}

Conforme \cite{spivak2014cts}, uma category consiste em uma coleção de coisas, todas relacionadas de algum modo. As coisas são nomeadas de objetos e as relações de morfismos.\\

\textbf{Definição}: a category $C$ é definida como:

\begin{enumerate}[label=(\Alph*), leftmargin=4em, topsep=0pt,itemsep=1ex,partopsep=1ex,parsep=1ex]
  \item uma coleção $Ob(C)$, contendo os objetos de $C$;
  \item para cada par $x, y \in Ob(C)$, um conjunto $Hom_C(x, y)$ chamado de morfismos de de $x$ para $y$;
\end{enumerate}

\subsection{Functor -- Categories e Morfismos}

Um functor $F$ consiste em um morfismo entre duas categories \cite{maclane1971mat}. Estes morfismos, chamados homomorfismos, tem como característica a preservação de estrutura de $F$. Assim como categories, functors devem satisfazer os axiomas de associatividade e identidade.\\

\textcolor{red}{<adicionar notações matemáticas e diagramas>}

\subsection{Monad -- Endofunctor e Transformações Naturais}

Endofunctors equipados com as natural transformations unit e multiplication \cite{maclane1971mat} são chamados de monad. Define-se endofunctors como functors cujos morfismos dão-se em uma mesma category. Além disso, natural transformations são morfismos entre functors, tendo como principal característica a preservação de estrutura. Como todas as estruturas já mencionadas, monads também devem satisfazer as propriedades de associatividade e identidade.\\

\textcolor{red}{<adicionar notações matemáticas e diagramas comutativos>}

\subsection{Arrow -- Kleisli Triples}

A partir de um monad podemos construir kleisli triples. Podemos pensar em kleisli triples como uma outra forma de representar sintaticamente um monad. Similar a monads, estas também obedecem as propriedades de associatividade e comutatividade.\\

\textcolor{red}{<adicionar notações matemáticas e diagramas comutativos>}

\bibliographystyle{IEEEtran}
\bibliography{references}

\end{document}