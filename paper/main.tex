\documentclass[10pt, conference]{IEEEtran}

\usepackage{cite}
\usepackage{listings}
\usepackage{lipsum}
\usepackage[T1]{fontenc}
\usepackage[portuguese]{babel}
\usepackage[utf8]{inputenc}

\begin{document}

\title { 
	Category Theory \\
	Arrows e Monads \\
	em Java 
}

\author{

	\IEEEauthorblockN{ Mozart L. Siqueira }
	\IEEEauthorblockA{
		Ciências da Computação \\
		Centro Universitário La Salle - Unilasalle \\
		Email: mozarts@unilasalle.edu.br
	}

	\and	
	
	\IEEEauthorblockN { Pablo M. Parada }
	\IEEEauthorblockA { 
		Ciências da Computação \\
		Centro Universitário La Salle - Unilasalle \\
		Email: pablo.paradabol@gmail.com
	}
}
					  
\maketitle

\begin{abstract}

<escrever>

\end{abstract}

\begin{IEEEkeywords}

<escrever>

\end{IEEEkeywords}

\section{INTRODUÇÃO}
\label{sec:intro}

<escrever>

\section{SOBRE O PARADIGMA FUNCIONAL}
\label{sec:func-para}
Influenciado principalmente pelo desenvolvimento do \textit{lambuda calculus} \cite{hudak1989conception}, compondo o grupo da programação declarativa, o paradigma funcional utiliza-se da idéia de expressar computações através de funções combinadas em expressões. Neste, funções expressam o que deverá ser computado, ao invés de como sera computado \cite{louden2011programming}. Programas são construídos através da composição, tal que funções triviais (ou \textit{building blocks}) são combinadas dando origem a novas funções que descrevem computações mais complexas.

\textit{Building blocks} não devem fazer uso de variáveis que dependam de estado, isso significa que a computação deve ser pura e sem efeitos indesejados (ou \textit{side-effects}). Também destaca-se o princípio de imutabilidade, onde o valor é de uma variável é determinado em sua criação, não permitindo novas atribuições posteriormente.

É possível afirmar que ao expressar um programa em uma linguagem funcional, obtem-se uma maneira concisa de solucionar problemas, dado que este constitui-se de operações e objetos atômicos e regras gerais para sua composição \cite{michaelson2011introduction}. Estas qualidades são apreciadas nos tempos atuais, onde há necessidade de tratar os problemas oriundos do não-determinismo. Assim, o paradigma funcional mostra-se capaz, inclusive de influenciar outras linguagens como \textit{Java} \cite{jsr335}.

\subsection{Anonymous Inner Classes e Lambda Expressions}
A cada release cycle, Java incrementa seu ambiente de desenvolvimento fornecendo novos recursos para seus desenvolvedores. Em sua oitava distribuição, a linguagem adicionou as funcionalidades necessárias para habilitar o uso de funções de primeira classe (ou \textit{lambda expressions}). 

AICs fornecem instruções básicas para a criação de representações concretas de interfaces e classes abstratas. Quando instanciadas, devem conter a implementação definida pelo contrato de sua interface. Estas podem também referenciar o objeto corrente utilizando a palavra reservada \textit{this}, possibilitando a invocação de métodos e a mutação de variáveis.

\bibliographystyle{IEEEtran}
\bibliography{references}

\end{document}